%----以下文字の大きさ,フォント,余白などを決定する設定----
\documentclass[a4paper,10pt]{jsarticle}
\setlength{\topmargin}{-17truemm}
\setlength{\oddsidemargin}{-0.4truemm}
\setlength{\textwidth}{160truemm}
\setlength{\textheight}{247truemm}
%-------------------------------------------------
%コンパイル方法 platex hoge.tex → platex hoge.tex → dvipdfmx hoge(サブファイルでも実行可能※ただし,サブファイルでコンパイルすると「他のサブファイルで定義された」ラベルを参照した時に?になる)
%bibtex適用のコンパイル platex hoge.tex → pbibtex hoge.tex → platex hoge.tex → platex hoge.tex → dvipdfmx hoge(メインファイルでのみ実行可能※サブファイルではコンパイルエラーが起きる)
%\usepackage{}:Latexで拡張機能を使用するための宣言-------------
\usepackage{longtable}%ページをまたぐ表を表示させるパッケージ
\usepackage{float}%その場に表・図を表示させるパッケージ
\usepackage{tabularx}%表内セルを良いところで改行

\usepackage[dvipdfmx]{hyperref}%しおり作成のためのパッケージ
\usepackage{pxjahyper}%しおりを日本語化

\title{タイトル}%タイトル
\author{田中太郎}%著者
\date{2000年1月1日}%日付

\begin{document}%本文の始まり
\maketitle%上記で設定したtitle,author,dateを表示

\section{表}
表\ref{tb:hoge}を参照します.
\begin{table}[H]
    \caption{hoge}
    \label{tb:hoge}
    \centering
    \begin{tabular}{ll}
        \hline
        title1 & title2 \\
        \hline \hline
        hoge1 & hoge2 \\
        \hline
    \end{tabular}
\end{table}

\end{document}

%表-------------------------
%\begin{table}[H]
%    \caption{hoge}
%    \label{tb:hoge}
%    \centering
%    \begin{tabular}{ll}
%        \hline
%        title1 & title2 \\
%        \hline \hline
%        hoge1 & hoge2 \\
%        \hline
%    \end{tabular}
%\end{table}
