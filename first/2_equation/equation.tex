%----以下文字の大きさ,フォント,余白などを決定する設定----
\documentclass[a4paper,10pt]{jsarticle}
\setlength{\topmargin}{-17truemm}
\setlength{\oddsidemargin}{-0.4truemm}
\setlength{\textwidth}{160truemm}
\setlength{\textheight}{247truemm}
%-------------------------------------------------
%コンパイル方法 platex hoge.tex → platex hoge.tex → dvipdfmx hoge(サブファイルでも実行可能※ただし,サブファイルでコンパイルすると「他のサブファイルで定義された」ラベルを参照した時に?になる)
%bibtex適用のコンパイル platex hoge.tex → pbibtex hoge.tex → platex hoge.tex → platex hoge.tex → dvipdfmx hoge(メインファイルでのみ実行可能※サブファイルではコンパイルエラーが起きる)
%\usepackage{}:Latexで拡張機能を使用するための宣言-------------
\usepackage{amsmath}%数式関連のパッケージ(alignなど)
\usepackage{bm}%太字でベクトルを表記

\usepackage[dvipdfmx]{hyperref}%しおり作成のためのパッケージ
\usepackage{pxjahyper}%しおりを日本語化

\title{タイトル}%タイトル
\author{田中太郎}%著者
\date{2000年1月1日}%日付

\begin{document}%本文の始まり
\maketitle%上記で設定したtitle,author,dateを表示

\section{数式}

式(\ref{eq:hoge})を参照します.

文章内で $y = x$ のように数式を記述することも可能です.

\begin{gather}
    x_{1}=L_{1} \cos \theta_{1}
    \label{eq:hoge}
\end{gather}

\end{document}

%-------ラベル,参照----------
%\label{eq:hoge}

%公式------------------
%\begin{gather}
%    x_{1}=L_{1} \cos \theta_{1}
%    \label{eq:hoge}
%\end{gather}

%式--------------------
%$x_1$

