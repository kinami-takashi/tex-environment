%----以下文字の大きさ,フォント,余白などを決定する設定----
\documentclass[a4paper,10pt]{jsarticle}
\setlength{\topmargin}{-17truemm}
\setlength{\oddsidemargin}{-0.4truemm}
\setlength{\textwidth}{160truemm}
\setlength{\textheight}{247truemm}
%-------------------------------------------------
%コンパイル方法 platex hoge.tex → platex hoge.tex → dvipdfmx hoge(サブファイルでも実行可能※ただし,サブファイルでコンパイルすると「他のサブファイルで定義された」ラベルを参照した時に?になる)
%bibtex適用のコンパイル platex hoge.tex → pbibtex hoge.tex → platex hoge.tex → platex hoge.tex → dvipdfmx hoge(メインファイルでのみ実行可能※サブファイルではコンパイルエラーが起きる)
%\usepackage{}:Latexで拡張機能を使用するための宣言-------------
\usepackage[dvipdfmx]{hyperref}%しおり作成のためのパッケージ
\usepackage{pxjahyper}%しおりを日本語化

\title{タイトル}%タイトル
\author{田中太郎}%著者
\date{2000年1月1日}%日付

\begin{document}%本文の始まり
\maketitle%上記で設定したtitle,author,dateを表示

\section{参考文献を参照}
PointNet\cite{point_net}を参照する.

%参考文献(bibtex(参考文献を反映させるときはpbibtexコマンドを使用 -----------------
% \nocite{*} %http://regry358.hatenablog.com/entry/2014/10/16/140251 参照していない文献も表示
\bibliographystyle{jIEEEtran.bst} % bstファイル(参考文献の参照方法を記述したファイル)にしたがって参考文献を記述
\bibliography{reference.bib} % bibファイル(参考文献の一覧)を参照
% https://github.com/ehki/jIEEEtran
% IEEJtran.bst : 電気学会の日英両対応版
% jIEEJtran.bst : IEEEtranの日本語対応版 (元ファイルから若干編集)
% mixej.py : 同一文献で日本語と英語を併記するためのスクリプト
% -------------------------------------------------------------------------
\end{document}

%参考文献,サイトの参照-----
%\cite{motion-planning}